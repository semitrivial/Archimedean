\documentclass[reqno]{article}
\usepackage{amsthm}
\usepackage{amsfonts}
\usepackage{amsmath}
\usepackage{mathrsfs}
\usepackage{amssymb}
\theoremstyle{definition}
\newtheorem{theorem}{Theorem}
\newtheorem{lemma}[theorem]{Lemma}
\newtheorem{question}[theorem]{Question}
\newtheorem{example}[theorem]{Example}
\newtheorem{proposition}[theorem]{Proposition}
\newtheorem{corollary}[theorem]{Corollary}
\newtheorem{definition}{Definition}

\begin{document}

\title{Archimedean measurements considered poor for non-Archimedean structures}

\author{Samuel A.~Alexander\thanks{Email:
samuelallenalexander@gmail.com}\hphantom{*}\footnote{2010 Mathematics 
Subject Classification: FillThisIn
(Primary), FillThisIn (Secondary)}\\
\emph{The U.S.\ Securities and Exchange Commission}}
\date{2020}
\maketitle

\begin{abstract}
    Fill this in.
\end{abstract}

\section{Introduction}

Whenever we measure anything using a particular number system, the
corresponding measurements will be constrained by the structure of that
number system. If the number system has a different structure than
the things we are measuring with it, then our
measurements will suffer accordingly, just as if we were trying to
force square pegs into round holes.

For example, the natural numbers make lousy candidates for measuring
distances within a physics laboratory. Distances in the lab have
properties such as, for example, the fact that for any two distinct
distances, there is an intermediate distance strictly between them.
The natural numbers lack this property. Imagine the poor physicist,
brought up in a world of only natural numbers, scratching his or her
head upon encountering a rod with length strictly between two rods
of length $1$ and $2$.

It's tempting to think of the real numbers $\mathbb R$---i.e., the unique
complete ordered field---as a generic number system with whatever
structure suits our needs. But it's important to remember that the
real numbers do have a very specific structure. That structure is
flexible enough to accomodate many needs, but we shouldn't just
take that for granted. One particular property of the real numbers
is the following.

\begin{lemma}
\label{archimedeanlemma}
(The Archimedean Property of the Real Numbers)
If $x_1,x_2,\ldots$ is an infinite sequence of real numbers,
where each $x_{i+1}$ is significantly larger than $x_i$---by which
we mean that each $x_{i+1}\geq x_i+1$---then, for every real number $y$,
there exists some $i$ such that $x_i>y$.
\end{lemma}

\begin{proof}
If not, then there is some $y$ such that $y\geq x_i$ for all $i$.
Thus, $X=\{x_1,x_2,\ldots\}$ has an upper bound. By the completeness
of $\mathbb R$, $X$ must have a least upper bound $z\in\mathbb R$.
Since $z$ is the least upper bound for $X$, $z-1$ is not an upper bound
for $X$, so there is some $i$ such that $x_i>z-1$.
By assumption, $x_{i+1}>x_i+1$, so $x_{i+1}>z$, contradicting the choice
of $z$.
\end{proof}

Note that the above property is automatically inherited by subsystems
of the reals, such as the rational numbers $\mathbb Q$, the natural
numbers $\mathbb N$, the integers $\mathbb Z$, or the algebraic numbers.
All inherit the Archimedean property in the obvious way.

Suppose we have some fuzzy widgets, and we observe that certain
widgets are fuzzier than others. Naturally, we are inclined to
quantify the fuzziness of the widgets, assigning them numerical
fuzziness measures from some number system. Nine times out of ten,
we choose to use the real numbers, or a subsystem of the real numbers,
often without a second thought. But suppose
among these widgets, there happen to be widgets $w_1,w_2,\ldots$
such that each $w_{i+1}$ is significantly fuzzier than $w_i$,
and another widget $w_\infty$ which is fuzzier than all the $w_i$'s.
Suddenly, our decision to use real numbers puts us in a bind.
It is impossible to assign real number fuzziness measures to our
widgets in such a way that fuzzier widgets get larger real number
fuzziness measures and such that significantly fuzzier widgets get
significantly larger real number fuziness measures. That would
contradict Lemma \ref{archimedeanlemma}.

The premise of this paper is that certain structures are inherently
non-Archimedean, and that when deciding how to measure such structures,
we should be aware of potential consequences of choosing an
Archimedean number system for those measurements.


\section{Background}

\section{First Example: Quality of graph-clustering algorithms}

\section{Second Example: Intelligence of strong AIs}

\section{Third Example: Reinforcement learning rewards}

\section{Conclusion}

\end{document}
