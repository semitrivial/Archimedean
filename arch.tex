\documentclass[reqno]{article}
\usepackage{amsthm}
\usepackage{amsfonts}
\usepackage{amsmath}
\usepackage{mathrsfs}
\usepackage{amssymb}
\theoremstyle{definition}
\newtheorem{theorem}{Theorem}
\newtheorem{lemma}[theorem]{Lemma}
\newtheorem{question}[theorem]{Question}
\newtheorem{example}[theorem]{Example}
\newtheorem{proposition}[theorem]{Proposition}
\newtheorem{corollary}[theorem]{Corollary}
\newtheorem{definition}{Definition}

\begin{document}

\title{Why traditional machine learning will not lead to AGI:
Archimedean measurements considered poor for non-Archimedean structures}

\author{Samuel A.~Alexander\thanks{Email:
samuelallenalexander@gmail.com}\hphantom{*}\footnote{2010 Mathematics 
Subject Classification: FillThisIn
(Primary), FillThisIn (Secondary)}\\
\emph{The U.S.\ Securities and Exchange Commission}}
\date{2020}
\maketitle

\begin{abstract}
    Fill this in.
\end{abstract}

\section{Introduction}

Whenever we measure anything using a particular number system, the
corresponding measurements will be constrained by the structure of that
number system. If the number system has a different structure than
the things we are measuring with it, then our
measurements will suffer accordingly, just as if we were trying to
force square pegs into round holes.

For example, the natural numbers make lousy candidates for measuring
distances in a physics laboratory. Distances in the lab have
properties such as, for example, the fact that for any two distinct
distances, there is an intermediate distance strictly between them.
The natural numbers lack this property. Imagine the poor physicist,
brought up in a world of only natural numbers, scratching his or her
head upon encountering a rod with length strictly between two rods
of length $1$ and $2$.

It's tempting to think of the real numbers $\mathbb R$---i.e., the unique
complete ordered field---as a generic number system with whatever
structure suits our needs. But it's important to remember that the
real numbers do have a very specific structure. That structure is
flexible enough to accomodate many needs, but we shouldn't just
take that for granted. One particular property of the real numbers
is the following.

\begin{lemma}
\label{specializedarchimedeanlemma}
(The Archimedean Property)
Let $r>0$ be any positive real number.
For every real number $y$, there is some natural number $n$
such that $nr>y$.
\end{lemma}

Rather than directly prove Lemma \ref{specializedarchimedeanlemma},
we will prove a generalized result more adaptable
to other structures.

\begin{lemma}
\label{generalizedarchimedeanlemma}
(The Generalized Archimedean Property)
Let $r>0$ be any positive real number.
If $x_0,x_1,x_2,\ldots$ is any infinite sequence of real numbers,
where each $x_{i+1}$ is significantly larger than $x_i$---by which
we mean that each $x_{i+1}\geq x_i+r$---then, for every real number $y$,
there exists some $i$ such that $x_i>y$.
\end{lemma}

\begin{proof}
If not, then there is some $y$ such that $y\geq x_i$ for all $i$.
Thus, $X=\{x_0,x_1,x_2,\ldots\}$ has an upper bound. By the completeness
of $\mathbb R$, $X$ must have a least upper bound $z\in\mathbb R$.
Since $z$ is the least upper bound for $X$, $z-r$ is not an upper bound
for $X$, so there is some $i$ such that $x_i>z-r$.
By assumption, $x_{i+1}>x_i+r$, so $x_{i+1}>z$, contradicting the choice
of $z$.
\end{proof}

Lemma \ref{specializedarchimedeanlemma} follows from
Lemma \ref{generalizedarchimedeanlemma} by letting $x_i=ir$.

Note that the above property is automatically inherited by subsystems
of the reals, such as the rational numbers $\mathbb Q$, the natural
numbers $\mathbb N$, the integers $\mathbb Z$, or the algebraic numbers.
All inherit the Generalized Archimedean Property in the obvious way.

Suppose we have some fuzzy widgets, and we observe that certain
widgets are fuzzier than others. Naturally, we are inclined to
quantify the fuzziness of the widgets, assigning them numerical
fuzziness measures from some number system. Nine times out of ten,
we choose to use the real numbers, or a subsystem of the real numbers,
often without a second thought. But suppose
among these widgets, there happen to be widgets $w_1,w_2,\ldots$
such that each $w_{i+1}$ is significantly fuzzier than $w_i$,
and another widget $w_\infty$ which is fuzzier than all the $w_i$'s.
Suddenly, our decision to use real numbers puts us in a bind.
It is impossible to assign real number fuzziness measures to our
widgets in such a way that fuzzier widgets get larger real number
fuzziness measures and such that significantly fuzzier widgets get
significantly larger real number fuziness measures. That would
contradict Lemma \ref{generalizedarchimedeanlemma}.

The premise of this paper is that certain structures are inherently
non-Archimedean, and that when deciding how to measure such structures,
we should be aware of potential consequences of choosing an
Archimedean number system for those measurements.


\section{Background and Preliminaries}

The Archimedean property is named after Archimedes of Syracuse.
It appears as the fifth axiom in his \emph{On the Sphere
and Cylinder} \cite{archimedes}:
\begin{quote}
    Further, of unequal lines, unequal surfaces, and unequal
    solids, the greater exceeds the less by such a magnitude
    as, when added to itself, can be made to exceed any
    assigned magnitude among those which are comparable with
    [it and with] one another.
\end{quote}
The Archimedean property is also closely related to Definition 4 of Book V of Euclid's
\emph{Elements} \cite{euclid}:
\begin{quote}
    (Those) magnitudes are said to have a ratio
    with respect to one another which, being
    multiplied, are capable of exceeding one
    another.
\end{quote}

The real numbers possess the Archimedean property, but other structures
may or may not possess the property. In order to make this more precise,
we introduce the following formalism, adapting from Lemma \ref{generalizedarchimedeanlemma}.

\begin{definition}
    A \emph{significantly ordered structure} is a set $X$ along with
    two orderings $<$ and $\ll$.
    For $x,y\in X$, we say $y$ is \emph{greater} than $x$ if $x<y$, and
    we say $y$ is \emph{significantly greater} than $x$ if $x\ll y$.
    A significantly ordered structure is said to be \emph{Archimedean} if it
    has the following property: for every $X$-sequence
    $x_0\ll x_1\ll x_2 \ll \cdots$,
    for every $x_\infty\in X$, there is some $i$ such that $x_\infty<x_i$.
\end{definition}

The prototypical example of an Archimedean significantly ordered structure is the real
numbers with the usual ordering $<$ and the significantly-less-than
ordering defined such that $x\ll y$ if $y\geq x+1$.
Before giving some examples of significantly ordered structures that are non-Archimedean,
we will first develop some more formalism.

\begin{definition}
    Suppose $(X,<_X,\ll_X)$ and $(Y,<_Y,\ll_Y)$
    are two significantly ordered structures.
    A function $f:X\to Y$ is \emph{order-preserving} if it satisfies the following
    requirements:
    \begin{enumerate}
        \item
            For all $x_1,x_2\in X$, $x_1<_X x_2$ if and only if $f(x_1)<_Yf(x_2)$.
        \item
            For all $x_1,x_2\in X$, $x_1\ll_X x_2$ if and only if
            $f(x_1)\ll_Yf(x_2)$.
    \end{enumerate}
\end{definition}

The following proposition formalizes the dilemma we illustrated with the
example of fuzzy widgets in the Introduction (think of $X$ as things we
want to measure, and think of $Y$ as a number system we want to use to
perform said measurement).

\begin{proposition}
\label{maindilemma}
(Inadequacy of Archimedean number systems for non-Archimedean structures)
    Suppose $(X,<_X,\ll_X)$ and $(Y,<_Y,\ll_Y)$
    are significantly ordered structures.
    If $Y$ is Archimedean and $X$ is non-Archimedean, then there cannot exist
    an order-preserving function $f:X\to Y$.
\end{proposition}

\begin{proof}
    Assume, for sake of a contradiction, that there is some order-preserving
    $f:X\to Y$.
    Since $X$ is non-Archimedean, there is some $X$-sequence
    $x_0\ll_X x_1\ll_X \cdots$ and some
    $x_\infty\in X$ such that there is no $i$ such that $x_\infty<_X x_i$.
    Since $f$ is order-preserving,
    $f(x_0)\ll_Y f(x_1) \ll_Y \cdots$.
    Since $Y$ is Archimedean, there must be some $i$ such that
    $f(x_\infty)<_Y f(x_i)$.
    Since $f$ is order-preserving, $x_\infty<_X x_i$, a contradiction.
\end{proof}

Proposition \ref{maindilemma} tells us that we cannot faithfully measure
non-Archimedean structures using an Archimedean number system.
Any attempt to do so will necessarily be misleading, because ordering
relationships among the non-Archimedean structures will fail to be reflected
by the numerical measurements given to them.
We will inevitably find ourselves in the shoes of the puzzled physicist
brought up in a world of only natural numbers, confronted by a rod of
length $1.5$.

\begin{example}
(Examples of non-Archimedean structures)
    \begin{itemize}
        \item
        (Probabilities)
        The measurable subsets of $\mathbb R$ are non-Archimedean
        if we define $<$ and $\ll$ such that for all measurable subsets
        $X,Y$ of $\mathbb R$, $X<Y$ iff $X$ is a subset of $Y$
        and $X\ll Y$ if and only if $X$ is a subset of $Y$ and $|Y-X|=\infty$.
        Nevertheless, probability theorists routinely
        measure such sets using the Archimedean real number system.
        The inadequacy expressed in Proposition \ref{maindilemma}
        manifests itself in well-known paradoxes such as the fact that
        an event can have probability $0$ and yet still be possible,
        or can have probabity $1$ and yet not be guaranteed,
        or that an infinite event can have probability $0$.
        See \cite{benci2013non} for an alternative non-Archimedean approach
        to probability.
    \end{itemize}
\end{example}

\section{First Example: Reinforcement learning rewards}

\section{Second Example: Unsupervised learning}

\section{Third Example: Supervised learning}

\section{Fourth Example: Intelligence of AGIs}

\begin{quote}
    ``The very day you start, you will go home a better man,
    and the same thing will happen the day after. Every day,
    day after day, you will get better and better.''
    (Protagoras, as quoted by Socrates, as quoted by Plato \cite{protagoras})
\end{quote}

\section{Conclusion}

\bibliographystyle{plain}
\bibliography{arch}
\end{document}
