\documentclass[reqno]{article}
\usepackage{amsthm}
\usepackage{amsfonts}
\usepackage{amsmath}
\usepackage{mathrsfs}
\usepackage{amssymb}
\theoremstyle{definition}
\newtheorem{theorem}{Theorem}
\newtheorem{lemma}[theorem]{Lemma}
\newtheorem{question}[theorem]{Question}
\newtheorem{example}[theorem]{Example}
\newtheorem{remark}[theorem]{Remark}
\newtheorem{proposition}[theorem]{Proposition}
\newtheorem{corollary}[theorem]{Corollary}
\newtheorem{definition}{Definition}

\begin{document}

\title{Traditional reinforcement learning will probably not yield AGI:
Limitations of Archimedean rewards}

\author{Samuel A.~Alexander\thanks{Email:
samuelallenalexander@gmail.com}\hphantom{*}\footnote{2010 Mathematics 
Subject Classification: 97R40 (Primary), 26E35 (Secondary)}\\
\emph{The U.S.\ Securities and Exchange Commission}}
\date{2020}
\maketitle

\begin{abstract}
    After generalizing the Archimedean property of real numbers in such a
    way as to make it adaptable to non-numeric structures, we demonstrate
    that the real numbers cannot be used to accurately measure non-Archimedean
    structures. We argue that, since an agent with Artificial General
    Intelligence (AGI) should have no problem engaging in tasks that inherently
    involve non-Archimedean rewards, and since traditional reinforcement
    learning rewards are real numbers, therefore traditional reinforcement
    learning cannot lead to AGI. We indicate two possible ways traditional
    reinforcement learning could be altered to remove this roadblock.
\end{abstract}

\section{Introduction}

Whenever we measure anything using a particular number system, the
corresponding measurements will be constrained by the structure of that
number system. If the number system has a different structure than
the things we are measuring with it, then our
measurements will suffer accordingly, just as if we were trying to
force square pegs into round holes.

For example, the natural numbers make lousy candidates for measuring
distances in a physics laboratory. Distances in the lab have
properties such as, for example, the fact that for any two distinct
distances, there is an intermediate distance strictly between them.
The natural numbers lack this property. Imagine the poor physicist,
brought up in a world of only natural numbers, scratching his or her
head upon encountering a rod with length strictly between two rods
of length $1$ and $2$.

It's tempting to think of the real numbers $\mathbb R$---i.e., the unique
complete ordered field---as a generic number system with whatever
structure suits our needs\footnote{For example, Russell describes
measurement as ``any method by which a ...\ correspondence
is established between ...\ magnitudes of a kind and ...\ numbers,
integral, rational or real'' \cite{russell1903principles}. Russell's
credibility on this point
is cast into doubt by Bair et al \cite{bair2013mathematical}.}. But it's
important to remember that the
real numbers do have a very specific structure. That structure is
flexible enough to accomodate many needs, but we shouldn't just
take that for granted. One particular property of the real numbers
is the following.

\begin{lemma}
\label{specializedarchimedeanlemma}
(The Archimedean Property\footnote{The Archimedean property is named after Archimedes of Syracuse.
A similar property appears as the fifth axiom in his \emph{On the Sphere
and Cylinder} \cite{archimedes}:
\begin{quote}
    Further, of unequal lines, unequal surfaces, and unequal
    solids, the greater exceeds the less by such a magnitude
    as, when added to itself, can be made to exceed any
    assigned magnitude among those which are comparable with
    [it and with] one another.
\end{quote}
Note that Archimedes specifically restricts his statement to
lengths, surface areas and volumes, in fact going out of his
way to limit the magnitudes to which said length/area/volume
can be made to exceed (he could have saved some words
by stopping his sentence at ``...can be made to exceed any
assigned magnitude'', if that were his intention).

The Archimedean property is also closely related to Definition 4 of Book V of Euclid's
\emph{Elements} \cite{euclid}:
\begin{quote}
    (Those) magnitudes are said to have a ratio
    with respect to one another which, being
    multiplied, are capable of exceeding one
    another.
\end{quote}
Proposition 1 of Book X is also relevant.
Many math historians
speak of the modern-day Archimedean property, Archimedes' 5th axiom, and Euclid's
properties as being identical, but in fact they are all subtly different from one
another, see \cite{bair2013mathematical}.})
Let $r>0$ be any positive real number.
For every real number $y$, there is some natural number $n$
such that $nr>y$.
\end{lemma}

Rather than directly prove Lemma \ref{specializedarchimedeanlemma},
we will prove a generalized result which, we will argue, is
more adaptable to other structures.

\begin{lemma}
\label{generalizedarchimedeanlemma}
(The Generalized Archimedean Property)
Let $r>0$ be any positive real number.
For any $x,y\in\mathbb R$, say that $y$ is \emph{significantly greater}
than $x$ if $y\geq x+r$.
If $x_0,x_1,x_2,\ldots$ is any infinite sequence of real numbers,
where each $x_{i+1}$ is significantly greater than $x_i$, then for every real number $y$,
there exists some $i$ such that $x_i$ is significantly greater than $y$.
\end{lemma}

\begin{proof}
If not, then there is some $y$ such that $y+r > x_i$ for all $i$.
Thus, $X=\{x_0,x_1,x_2,\ldots\}$ has an upper bound. By the completeness
of $\mathbb R$, $X$ must have a least upper bound $z\in\mathbb R$.
Since $z$ is the least upper bound for $X$, $z-r$ is not an upper bound
for $X$, so there is some $i$ such that $x_i>z-r$.
By assumption, $x_{i+1}>x_i+r$, so $x_{i+1}>z$, contradicting the choice
of $z$.
\end{proof}

Lemma \ref{specializedarchimedeanlemma} follows from
Lemma \ref{generalizedarchimedeanlemma} by letting $x_i=ir$.

The above property is automatically inherited by subsystems
of the reals, such as the rational numbers $\mathbb Q$, the natural
numbers $\mathbb N$, the integers $\mathbb Z$, or the algebraic numbers.
All inherit the Generalized Archimedean Property in obvious ways.

Lemma \ref{generalizedarchimedeanlemma} allows us to adapt the notion
of Archimedeanness to other things than real numbers, even to things
for which there is no notion of arithmetic at all
(Lemma \ref{specializedarchimedeanlemma} would not adapt to such things).
All we need is a notion of ``significantly greater than''.
For any set of things, some of which are ``significantly greater than''
others, we can ask whether or not the property in Lemma
\ref{generalizedarchimedeanlemma} holds. We will make this formal in
Section \ref{backgroundsection}.

\begin{example}
\label{fuzzywidgets}
(Fuzzy widgets)
Suppose we have some fuzzy widgets, and we observe that certain
widgets are fuzzier than others. Naturally, we are inclined to
quantify the fuzziness of the widgets, assigning them numerical
fuzziness measures from some number system. Nine times out of ten,
we choose to use the real numbers, or a subsystem thereof,
often without a second thought. But suppose
among these widgets, there happen to be widgets $w_1,w_2,\ldots$
such that each $w_{i+1}$ is significantly fuzzier than $w_i$,
and another widget $w_\infty$ which is significantly fuzzier than all the $w_i$'s.
Suddenly, our decision to use real numbers puts us in a bind.
It is impossible to assign real number fuzziness measures to our
widgets in such a way that significantly fuzzier widgets get
significantly greater real number measures. That would
contradict Lemma \ref{generalizedarchimedeanlemma}.
\end{example}

Note that the above example does not require us to have any notion
of multiplying fuzziness by a natural number $n$ (as we would need to have
if we wanted to adapt Lemma \ref{specializedarchimedeanlemma}).
This illustrates the enhanced adaptability of Lemma \ref{generalizedarchimedeanlemma}.

The structure of this paper is as follows.
\begin{itemize}
    \item
    In Section \ref{backgroundsection} we formally adapt
    Lemma \ref{generalizedarchimedeanlemma} to obtain a notion of Archimedeanness
    for non-numerical structures, and demonstrate that non-Archimedean such
    structures cannot accurately be measured using the real numbers.
    \item
    In Section \ref{reinforcementlearningsection} we argue that
    traditional reinforcement learning will not lead to AGI because
    its rewards are overly constrained.
    \item
    In Section \ref{nontraditionalsection} we discuss non-traditional
    variations of reinforcement learning that avoid the problem of
    overly constrained rewards.
    \item
    In Section \ref{conclusionsection} we summarize and make concluding remarks.
\end{itemize}


\section{Generalized Archimedean Structures}
\label{backgroundsection}

The real numbers possess the Archimedean property, but other structures
may or may not. To make this more precise,
we introduce the following formalism, adapting from Lemma \ref{generalizedarchimedeanlemma}.

\begin{definition}
    A \emph{significantly-ordered structure} is a collection $X$ with
    an ordering $\ll$.
    For $x_1,x_2\in X$, we say $x_2$ is \emph{significantly greater}
    than $x_1$ if $x_1\ll x_2$.
    A significantly-ordered structure is \emph{Archimedean} if it
    has the following property: for every $X$-sequence
    $x_0\ll x_1\ll x_2 \ll \cdots$,
    for every $x_\infty\in X$, there is some $i$ such that $x_\infty\ll x_i$.
\end{definition}

For any real number $r>0$, a prototypical example of an Archimedean
significantly-ordered structure is the real
numbers with $\ll$ defined such that
$x\ll y$ if and only if $y\geq x+r$.

\begin{definition}
    Suppose $(X,\ll)$ is a significantly-ordered structure.
    A function $f:X\to\mathbb R$ is said to \emph{respect $\ll$}
    if there is some real $r>0$ such that the following requirement holds:
    \begin{itemize}
        \item
        For all $x,y\in X$, $x\ll y$ if and only if
        $f(y)\geq f(x)+r$.
    \end{itemize}
\end{definition}

The following proposition formalizes the dilemma we illustrated in
Example \ref{fuzzywidgets} (think of $X$ as a set of things we
want to measure).

\begin{proposition}
\label{maindilemma}
(Inadequacy of the reals for non-Archimedean structures)
    Suppose $(X,\ll)$ is a significantly-ordered structure.
    If $X$ is non-Archimedean, then no function $f:X\to\mathbb R$
    respects $\ll$.
\end{proposition}

\begin{proof}
    Assume, for sake of a contradiction, that some $f:X\to\mathbb R$
    exists which respects $\ll$. Thus there is some real $r>0$ such that
    for all $x,y\in X$, $x\ll y$ if and only if $f(y)\geq f(x)+r$.
    Since $X$ is non-Archimedean, there is some $X$-sequence
    $x_0\ll x_1\ll x_2\ll\cdots$ and some $x_\infty\in X$
    such that there is no $i$ such that $x_\infty\ll x_i$.
    By choice of $r$, each $f(x_{i+1})\geq f(x_i)+r$ and there is no
    $i$ such that $f(x_i)\geq f(x_\infty)$.
    This contradicts Lemma \ref{generalizedarchimedeanlemma}.
\end{proof}

Proposition \ref{maindilemma} tells us that we cannot faithfully measure
non-Archimedean structures using real numbers\footnote{There is an area of research
known as \emph{measurement theory}, which, traditionally,
``takes the real numbers as a pre-given numerical domain'' \cite{niederee1992numbers}.
Some work has been done to generalize measurement theory away from this assumption
\cite{narens1974} \cite{skala1975} \cite{rizza2016divergent}.
We would submit this paper as further motivation in that direction.}.
Any attempt to do so will necessarily be misleading, because ordering
relationships among the non-Archimedean structures will fail to be reflected
by the real-number measurements given to them.
We will inevitably end up like the puzzled physicist
brought up in a world of only natural numbers, confronted by a rod of
length $1.5$.

\begin{remark}
(Spearman's Law of Diminishing Returns)
Suppose $(X,\ll)$ is a non-Archimedean significantly-ordered structure
with elements $x_0,x_1,\ldots$ and $x_\infty$ such that
$x_0\ll x_1\ll\cdots$ and each $x_i\ll x_\infty$.
Suppose $f:X\to\mathbb R$ has the property that $f(x_0)<f(x_1)<\cdots$
and each $f(x_i)<f(x_\infty)$. Then the monotone convergence theorem implies
that $\lim_{i\to\infty}f(x_i)$ converges, which in turn implies that
$\lim_{i\to\infty}(f(x_{i+1})-f(x_i))=0$. This suggests a general \emph{law
of diminishing returns}: any time a non-Archimedean significantly-ordered
structure $(X,\ll)$ is measured using
real numbers, if the measurement does not blatantly violate $\ll$ (in other
words, if there are no $x\ll y$ such that $x$ is given a larger real-number
measurement than $y$), then there will inevitably be elements $x_0\ll x_1\ll \cdots$
exhibiting \emph{diminishing returns}, in the sense that the measurements of
$x_i$ and $x_j$ are approximately equal for large enough $i,j$.
If human intelligence is non-Archimedean, then this could potentially explain
a psychometrical phenomenon called \emph{Spearman's Law of Diminishing Returns}
\cite{spearman1927abilities} \cite{blum2017spearman}
\cite{hernandez2019ai}, the empirical
tendency of cognitive ability tests to be less correlated in
higher-intelligence populations.
\end{remark}

\begin{example}
\label{nonexamples}
(Examples of non-Archimedean structures)
    \begin{itemize}
        \item
        (Sets)
        Say that set $U$ is significantly greater than set $V$ if there is an injective
        function from $V$ into $U$ but there is no bijective function from $V$
        onto $U$. It is easy to show there are sets $U_0,U_1,\ldots$, with each
        $U_{i+1}$ significantly greater than $U_i$, and $U_\infty=\cup_{i=0}^\infty U_i$
        is a set which is significantly greater than all of them. Thus, sets are
        non-Archimedean. In the field of \emph{set theory}, logicians measure
        the size of sets using Georg Cantor's famous non-Archimedean number system,
        the cardinal numbers.
        \item
        (Logical theories)
        It is not difficult to come up with (for example) true
        theories $T_0,T_1,\ldots$ (in the language of arithmetic) such that
        each $T_{i+1}$ proves the consistency of $T_i$, and an additional
        true theory $T_\infty$ (in the language of arithmetic)
        which proves the consistency of $\cup_{i=0}^\infty T_i$.
        In a sense, then, each $T_{i+1}$ is significantly stronger than $T_i$
        (see G\"odel's incompleteness theorems), and $T_\infty$ is
        significantly stronger than all of them. In this sense, logical theories
        are non-Archimedean. In the field
        of \emph{proof theory} \cite{pohlers2008proof},
        logicians measure the logical strength of theories not using real numbers
        but using computable ordinal numbers, another non-Archimedean number system.
        \item
        (Asymptotic runtime complexities)
        Suppose $A_0,A_1,\ldots$ are algorithms such that each $A_i$ has
        runtime complexity $\Theta(n^i)$, and suppose $A_\infty$ is an algorithm
        with runtime complexity $\Theta(2^n)$. Then in a certain sense, each
        $A_{i+1}$ has significantly higher asymptotic runtime complexity than $A_i$,
        and $A_\infty$ has significantly higher asymptotic runtime complexity than all
        the $A_i$'s. In this sense, asymptotic runtime complexity is non-Archimedean.
        In computer science, these runtime complexities are usually measured using
        big-$O$, big-$\Theta$, or similar notation systems.
    \end{itemize}
\end{example}

\begin{example}
\label{speculativeexamples}
    (Speculative examples of potentially non-Archimedean structures)
    Certain structures might plausibly be non-Archimedean, but it is a difficult
    question to say whether they truly are or not. The reader could come up with
    such examples in great abundance.
    \begin{itemize}
        \item
        (Musical beauty)
        Assuming there is such a thing as objective
        musical beauty (not contingent on features of the human condition, etc.),
        then it is plausible that music might be non-Archimedean, in the following
        sense: there might be songs $S_0,S_1,\ldots$ such that each $S_{i+1}$
        is significantly more beautiful than $S_i$, and another song
        $S_\infty$ which is significantly more beautiful than all the $S_i$'s.
        \item
        (Ethical utility)
        Early utilitarian Jeremy Bentham suggested a hedonistic
        calculus in which (presumably real) pleasure measurements would be assigned to
        actions, to help adjudicate ethical dilemmas.
        His successor, John Stuart Mill, objected that some actions are incomparably
        better than others: ``If one of [two pleasures] is, by those
        who are competently acquainted with both, placed so far above the other that
        they prefer it ...\ and would not resign it for any quantity of the other
        pleasure which their nature is capable of, we are justified in ascribing to
        the preferred enjoyment a superiority in quality, so far outweighing quantity
        as to render it, in comparison, of small account'' \cite{mill}.
        If $P_0$ and $P_\infty$ are pleasures such that
        each $P_{i+1}$ is significantly better than $P_i$ (where $P_i$ is the
        pleasure of ``experiencing $P_0$ $i$ times'') and such that $P_\infty$
        is significantly better than every $P_i$,
        then these utilitarian pleasures are non-Archimedean.
        \item
        (AGI)
        It is plausible that there
        are\footnote{As hinted by Protagoras, assuming Protagoras's own intelligence
        stays constant and remains higher than the intelligence of his student
        and that they live forever and that \emph{better} means \emph{significantly
        better}:
        ``The very day you start, you will go home a better man, and the same thing
        will happen the day after. Every day, day after day, you will get better
        and better.'' \cite{protagoras}} AGIs $A_0,A_1,\ldots$ such that
        each $A_{i+1}$ is significantly more
        intelligent than $A_i$, and another AGI $A_\infty$ which is significantly more
        intelligent than all the $A_i$'s. We first pointed this out in
        \cite{alexander2019measuring}, where we propose measuring the
        intelligence of mechanical
        knowing agents using computable ordinals, the same non-Archimedean number system
        which proof theorists use to measure logical strength of mathematical
        theories. Incidentally, if AGI intelligence is non-Archimedean, then
        Proposition \ref{maindilemma} shows it is
        impossible to measure machine intelligence using real numbers without some
        of those measurements being misleading\footnote{This would solve an open problem
        implicitly stated by Legg and Hutter \cite{legg} when they said of their
        real-number universal intelligence measure: ``...none of these people have
        been able to communicate why the work [on measuring universal intelligence
        using real numbers] is so obviously flawed in any concrete way ...
        If anyone would like to properly explain their position to us in the future,
        we promise not to chase you down the street!''}.
        \item
        (Nonstandard cosmologies)
        Some
        authors
        \cite{al2016surreal} \cite{andreka2012logic}
        \cite{reeder2012infinitesimals} \cite{rosinger2007cosmic}
        \cite{chen2019infinitesimal} have
        even speculated about the nature of non-Archimedean space and/or
        time.
    \end{itemize}
\end{example}


\section{Reinforcement learning}
\label{reinforcementlearningsection}

In reinforcement learning (RL), an agent interacts with an environment,
taking actions from a fixed set of possible actions. With every action the
agent takes, the environment responds with a new observation and with a
reward. In traditional RL, these rewards are real numbers (many
authors further constrain them to be rational numbers).

By restricting rewards to be real (or rational) numbers, we unconsciously
constrain RL to only be applicable toward tasks of an inherently Archimedean
nature. For example, Wirth et al point out \cite{wirth2017survey} that
in tasks related to cancer treatment \cite{zhao2009reinforcement},
``the death of a patient should be avoided at any cost. However, an
infinitely negative reward breaks classic reinforcement learning algorithms
and arbitrary, finite values have to be selected.'' This problem could be
avoided if instead of real numbers, rewards were drawn from a suitable
non-Archimedean number system containing negative infinities,
for example, from the so-called
surreal numbers (which we will discuss further on). Doing so would, however,
be a departure from traditional RL.

To give an intuitive example, assume that musical beauty is non-Archimedean,
as in Example \ref{speculativeexamples}. We can imagine environments where the
RL agent is tasked with composing songs. For example, the possible actions the
agent is allowed to take might include one action for each piano key, plus
an additional ``stand and bow'' action to signal that a song is
finished.
Whenever the agent stands and bows, the agent is awarded with applause based on
the beauty of the song the agent
composed\footnote{To quote Wang and Hammer: ``Decision makings often do not happen
at the level of basic operations, but at the level of composed actions, where
there are usually infinite possibilities.'' \cite{wang2015assumptions}}. Assuming
musical beauty is
non-Archimedean, such an environment falls outside the possibility of traditional
RL. By Lemma \ref{maindilemma}, there is no way to assign real number
rewards to songs without misleading the agent. If $S_0,S_1,S_2,\ldots$ are songs
where each $S_{i+1}$ is significantly more beautiful than $S_i$, and $S_\infty$
is a song significantly more beautiful than all the $S_i$'s, then there is no way to
assign real-valued rewards to these songs such that each $S_{i+1}$ gets
significantly more reward than $S_i$ and such that $S_\infty$ gets significantly
more reward than all the $S_i$'s.

Or, to re-use the cancer example, assume there are certain bad procedures the
robotic surgeon could take, each one significantly worse than the previous,
but all still significantly better than killing the patient. There is no way
to assign real-valued rewards to these actions, and to killing the patient,
in such a way that each bad action gets punished significantly harsher than
the previous, but still significantly more forgiving than the punishment for
killing the patient.

The reader might object by challenging the non-Archimedeanness
of music and of medical procedures. But we only used those to make the examples
more intuitive. If the reader insists, we can resort to mathematical tasks.
For example, imagine that instead of playing piano, the agent is tasked with
typing up mathematical theories, and when the agent stands and bows, the agent
is rewarded with applause based on the proof-theoretical strength of the theory
(or hit with tomatoes if the theory is inconsistent).
In Example \ref{nonexamples} we noted that proof-theoretical strength of theories
is non-Archimedean. There exist theories $T_0,T_1,\ldots$, each significantly
proof-theoretically stronger than the previous, and another theory $T_\infty$,
significantly proof-theoretically stronger than all the $T_i$'s. We cannot
possibly assign real-valued rewards to these theories without misleading the
agent.

The reader might object to the above example because judging the proof-theoretical
strength of a theory is inherently non-computable anyway. The example could be
modified so that instead of typing up mathematical theories, the agent has to
type up mathematical subtheories in (say) the language of Peano arithmetic,
accompanied by consistency proofs in (say) ZFC. It can be shown that the
proof-theoretical strength of mathematical theories is still non-Archimedean,
even when restricted to subtheories of arithmetic whose consistency can be
proven in ZFC.

The reader might object that the above theories-with-proofs example is contrived.
But if an AGIs has human or superhuman intelligence, then even such contrived
tasks should pose no problem for that AGI. When we prove that the Halting Problem
is unsolvable, we do so by considering contrived programs that we could write if
the Halting Problem were solvable. The contrivedness of those programs does not
invalidate the proof of the unsolvability of the Halting Problem. Again, when we
prove that C++ templates are Turing complete \cite{veldhuizen}, we do so by
considering extremely
bizarre C++ templates that would never arise naturally in a software
studio. This does not invalidate the proof that C++ templates are Turing complete.
Reinforcement Learning is very useful for many practical tasks, but at least in
its traditional flavor, it is too constrained (by its arbitrary choice of number
system for its rewards) to apply to certain
non-Archimedean tasks\footnote{Perhaps explaining why
``despite almost two decades of RL research, there has been little solid
evidence of RL systems that may one day lead to [AGI]''
\cite{livingston}.}, which, however contrived they are, could certainly be
attempted by an AGI. Traditional reinforcement learning will not lead to AGI.

\section{Non-traditional reinforcement learning}
\label{nontraditionalsection}

We have argued that traditional RL cannot lead to AGI, because
an AGI is capable of attempting non-Archimedean tasks whose rewards are
too rich to be expressed using real numbers. There are at least two
potential ways to change RL so as to make it applicable to such tasks and,
thus, at least potentially capable of leading to AGI.

\subsection{Preference-based reinforcement learning}

A lot of exciting research has been done on non-traditional variations
of RL where, instead of giving the agent numerical rewards for taking actions,
one instead informs the agent about the relative preference of various
actions or action-sequences. See \cite{wirth2017survey} for a survey.
This very nicely side-steps the problems from this paper.

% A problem still does remain, as illustrated by the cancer example (ibid).
% In traditional RL, although we would want to assign $-\infty$ reward to
% actions which kill the patient, the traditional RL number system forbids
% doing so (only real numbers may be rewards). Thus, one has to choose some
% arbitrary large finite negative number as the reward for killing the patient.
% This is egregious because, by the Archimedean property of the reals,
% it is possible that a long sequence of minor harmful actions could end up
% having a total negative reward even more negative than killing the patient.
% Preference-based machine learning solves the problem by allowing us to
% engineer preferences such that killing the patient will \emph{always} be
% considered relatively worse than any non-lethal action-sequence.
% But there remains the problem that this still does not convey the full
% gravity of the situation to the agent: at any particular time, the agent
% will know that patient-killing is worse than any non-lethal action-sequence
% observed so far,

\subsection{Reinforcement learning with other number systems}

The most obvious way to modify RL to avoid the problems presented in this
paper is to change which number system is used\footnote{Anticipated
by \cite{rizza2016divergent}.}.
As far as this author is aware,
the choice to use real (or rational) numbers for rewards was not made based
on any fundamental criteria\footnote{Nieder{\'e}e points out
\cite{niederee1992numbers} that there are
no deeper reasons to assume that the number system should necessarily
even have the same cardinality as $\mathbb R$. And Rizza says
\cite{rizza2016divergent}: ``No particular feature of the space of informational
states suggests that such a codomain [as $\mathbb R$] should be selected''.}. The real
(or rational) numbers are currently a
useful pragmatic choice because they are easy to compute with using 21st
century software and 21st century school curricula, but that's hardly relevant
in the field of genuine AGI. One might
say the real numbers were a good choice because they are familiar, but even
that is arguable: in general, students are not taught what the
real numbers \emph{actually are},
unless they major in pure mathematics at the university level. Anyway,
the familiarity argument is totally irrelevant in the field of AGI.

Various non-Archimedean number systems exist. Number systems can be
discrete or continuous; the nature of reinforcement learning clearly
suggests a continuous number system. We will consider three possible
number systems: formal Laurent series; hyperreal numbers; and
surreal numbers.

\subsubsection{Formal Laurent series}

David Tall described \cite{tall1980looking} described the following
real-number-extending number system (which he called the ``superreals'', but
that vocabular does not seem to have caught on).

\begin{definition}
A \emph{formal Laurent series} is a formal expression of the following form
(where $m$ can be any natural number and $a_{-m},a_{-(m-1)},\ldots,a_0,a_1,a_2,\ldots$
are real numbers, $a_{-m}\not=0$):
\[
a_{-m}x^{-m} + a_{-(m-1)}x^{-(m-1)} + \cdots + a_{-1}x^{-1} + a_0x^0 + a_1x^1 + a_2x^2 + \cdots
\]
\end{definition}

The advantage of the formal Laurent series number system is that it is
relatively concrete, compared to the more abstract hyperreal or surreal numbers
discussed below.

\begin{definition}
(Comparison of formal Laurent series)
Suppose $A=\sum_{i=-m}^\infty a_ix^i$
and $B=\sum_{i=-n}^\infty b_ix^i$ are two distinct formal Laurent series.
We declare $A<B$ if and only if $a_i<b_i$ where $i$ is the smallest
index such that $a_i\not=b_i$ (where we consider $a_i$ to be $0$ for all
$i<-m$ and we consider $b_i$ to be $0$ for all $i<-n$).
\end{definition}

\begin{example}
(Examples of formal Laurent series comparisons)
\begin{enumerate}
    \item Consider $A=5x^{-1}-2x^{0}+3x^1+4x^2$ and
    $B=5x^{-1}-2x^0+1x^1+4x^2+5x^6$.
    The $x^{-1}$- and $x^0$-coefficients of $A$ and $B$ are equal, so
    we compare their $x^1$-coefficients. $A$ has an $x^1$-coefficient of $3$ and
    $B$ has an $x^1$-coefficient of $1$, and $3>1$, so $A>B$.
    \item Consider $A=999999x^4$ and $B=0.00001x^{5}$.
    We consider $A$ to have an $x^{5}$-coefficient of $0$, which is smaller than
    $B$'s $x^{5}$-coefficient ($0.00001$), so $A<B$.
\end{enumerate}
\end{example}

There is a natural way to consider formal Laurent series as a significantly-ordered
structure, generalizing the notion of ``significantly greater than'' from
Lemma \ref{generalizedarchimedeanlemma}.

\begin{definition}
\label{significantorderednessoflaurent}
Let $r>0$ be any positive real number.
For any formal Laurent series $A=\sum_{i=-m}^\infty a_ix^i$
and $B=\sum_{i=-n}^\infty b_ix^i$,
we say $A\ll_r B$ if $A\not=B$ and $b_i\geq a_i+r$ where $i$ is the smallest
index such that $a_i\not=b_i$.
\end{definition}

\begin{lemma}
For any real $r>0$, the formal Laurent series, considered as a significantly-ordered
structure according to $\ll_r$, are non-Archimedean.
\end{lemma}

\begin{proof}
Let $x_0=0x^1$, $x_1=rx^1$, $x_2=2rx^1$, and in general let
$x_i=irx^1$. Let $x_\infty=1x^0$. Then each $x_i\ll_r x_{i+1}$,
and yet each $x_i\ll_r x_\infty$.
\end{proof}

We can consider the real numbers $\mathbb R$ to be embedded in the formal Laurent
series by way of the embedding $r\mapsto rx^0$. Having done so, the intuition is that,
for example, $1x^1$ is what we might call a ``first-order infinitesimal number'', smaller
than every positive real;
$1x^2$ is what we might call a ``second-order infinitesimal number'', smaller than every
positive first-order infinitesimal number; and so on. Likewise,
$1x^{-1}$ is what we might call a ``first-order infinite number'', bigger than
every real; $1x^{-2}$ is what we might call a ``second-order infinite number'', bigger
than every first-order infinite number; and so on. Thus, the formal Laurent series are
quite adequate to address the problem described by
Wirth et al \cite{wirth2017survey}, in which an infinite negative reward is required when the
RL agent kills the cancer patient.

Unfortunately, while the formal Laurent series contain infinities and infinitesimals,
in a sense we will make formal, they still do not contain ``enough'' infinities and
infinitesimals to accomodate genuine AGI. To make this formal, we introduce a weaker
notion of Archimedeanness.

\begin{definition}
Suppose $(X,\ll)$ is a significantly-ordered structure.
We define a new order $\ll'$ on $X$ as follows.
For any $x,y\in X$, we declare $x\ll' y$ if and only if there is a sequence
$x_0,x_1,\ldots$ such that the following conditions hold:
\begin{enumerate}
    \item
    $x_0=x$.
    \item
    Each $x_i\ll x_{i+1}$.
    \item
    Each $x_i\ll y$.
\end{enumerate}
We say $(X,\ll)$ is \emph{$2$-Archimedean} if for every $X$-sequence
$x_0\ll' x_1\ll' \cdots$ and every $y\in X$, there is some $i$ such that
$y\ll x_i$.
\end{definition}

\begin{lemma}
For any real $r>0$,
the formal Laurent series are $2$-Archimedean when considered as a
significantly-ordered structure as in Definition \ref{significantorderednessoflaurent}.
\end{lemma}

\begin{proof}
(To be filled in)
\end{proof}

Just as we argued that a genuine AGI should be capable of engaging in tasks that
involve inherently non-Archimedean rewards, by similar reasoning, an AGI should
be capable of engaging in tasks that involve inherently non-2-Archimedean rewards.
It is not hard to show that all the structures in Example \ref{nonexamples} are
non-2-Archimedean. Thus, replacing real number rewards by formal Laurent series
rewards is not enough to remove the roadblock, but it would at least expand the
types of rewards which are possible.


\subsubsection{Hyperreal numbers}

The field of mathematics where abstract calculus is done with infinite and infinitesimal
quantities is called \emph{nonstandard analysis} \cite{robinson}. The numbers most
commonly associated with this field are the so-called \emph{hyperreal numbers}.

The hyperreal numbers can be
introduced axiomatically or by means of a semi-constructive method which depends on
usage of a certain black box, a device known as a \emph{free ultrafilter}. Logicians
have proven that free ultrafilters exist but that, unfortunately, it is impossible to
concretely exhibit one. This severely limits (if not completely ruins) the practical
usefulness of reinforcement learning with hyperreal rewards.

Nevertheless, the hyperreals
might be useful for proving abstract structural properties about AGI\footnote{Similar to
the way we use free ultrafilters in \cite{alexander2019intelligence} to obtain
comparators of the utility-maximizing ability of traditional deterministic RL agents,
and prove structural properties about said comparators.
In fact, in that paper, we essentially independently re-invented the free ultrafilter
construction of the hyperreals, without realizing it at the time!}.
It can be shown that the hyperreals are not $2$-Archimedean, and indeed, not
$\alpha$-Archimedean for any ordinal $\alpha$, where ``$\alpha$-Archimedean'' refers to a
certain natural generalization of $2$-Archimedeanness. Thus, for the purpose of
proving abstract theorems about RL agents with generalized rewards, the hyperreals
would be more appropriate than the formal Laurent series.


\subsubsection{Surreal numbers}

All of the
well-known non-Archimedean extensions of $\mathbb R$
(including formal Laurent series and hyperreals) are subsystems of the
so-called \emph{surreal numbers}
\cite{conway} \cite{knuth} \cite{ehrlich2012absolute}. The surreal
numbers were initially discovered during John Conway's attempts to study
two-player combinatorial games like Go and Chess, so it would not be
surprising if they turn out to be important in the eventual development of
AGI.

% Calculus on the surreal numbers is still relatively cutting-edge, but work
% in this area in the last
% decade \cite{rubinstein2014surreal} \cite{lipparini2016surreal}
% \cite{costin2015integration}
% does seem to be
% moving in directions which could make gradient-descent methods possible over
% the surreals; perhaps someday it will be possible to do deep reinforcement learning
% with surreal rewards\footnote{Perhaps the overly-constrained rewards in traditional
% deep reinforcement learning offer an answer to at least the RL part of Goertzel's
% question \cite{goertzel2015there}, ``Are there Deep Reasons Underlying the
% Pathologies of Today's Deep Learning Algorithms?''
% (albeit a very different answer to the one Goertzel proposes)}. Perhaps
% work in that direction could be accelerated if
% researchers in the field of AGI would lend a hand.

Unlike the hyperreals,
the construction of the surreal numbers does not depend on any
non-constructive black boxes such as free ultrafilters.
They are constructed as the union of a hierarchy $S_\alpha$ of subsystems where
$\alpha$ ranges over the ordinal numbers. Assuming that agents with AGI are
implemented using computers with no additional power beyond the Church-Turing
Thesis, then for the purposes of AGI, it would be appropriate to restrict our
attention to a subset of computable surreal numbers, which is the union of
a hierarchy $C_\alpha$ where $\alpha$ ranges over the computable ordinal numbers.
For any particular level $C_\alpha$ in this hierarchy, we can consider the
sub-universe $E_\alpha$ of surreal-reward RL environments with rewards restricted
to $C_\alpha$.

Assuming the AGI agents are Turing computable,
no individual AGI can possibly comprehend codes for all computable ordinals, because
the set of codes of computable ordinals is badly non-computably-enumerable.
This is profound, because it seems to suggest that any particular AGI can
only comprehend RL environments in $E_\alpha$ if that AGI can comprehend $\alpha$.
In other words, for any particular RL environment $e$ with computable surreal number
rewards, there must be some minimal computable ordinal $\alpha$ such that $e$ has
rewards from $E_\alpha$; if an AGI is not intelligent enough to comprehend $\alpha$,
then it seems like there should be no way for the AGI to comprehend $e$
either\footnote{This situation is highly reminiscent of \cite{hibbard2011measuring}.}.
We would submit
this state of affairs as strong evidence in favor of our thesis
\cite{alexander2019measuring} that a machine's intelligence ought to be measured
in terms of the computable ordinals which the machine comprehends.

The above paragraph points at a possible joint path
toward AGI incorporating both machine learning and symbolic logic, perhaps
a much-needed reconciliation of these two approaches.

\subsection{Alternate number systems: tentative verdict}

For many simple environments not too far outside of traditional RL,
formal Laurent series could probably serve as a fairly practical number system.
But formal Laurent series have limitations which suggest that RL
with formal Laurent series rewards will not be enough to reach AGI, for the exact
same reason that RL with real number rewards will not be enough.

Because of their dependence on free ultrafilters, the hyperreal numbers will
probably never be of practical use as RL rewards, but it could conceivably be
possible to use them to prove abstract structural results about AGI from a
bird's-eye view.

The surreal numbers (or a computable subset thereof) seem like the most promising
candidate for RL rewards that could plausibly lead to AGI. We would certainly
hesitate to call them ``practical'', though. To work with any but the most trivial
of surreal numbers, one would need to implement sophisticated machinery for
dealing with computable ordinals, and that's just to get one's foot in the door.

For proving abstract structural results about AGI, it is unclear
whether hyperreal number RL rewards or surreal number RL rewards would be more
useful. It is plausible that certain abstract structural results could be easier
to prove using hyperreal rewards, and certain results could be easier to
prove using surreal rewards.


\section{Conclusion}
\label{conclusionsection}

In traditional reinforcement learning, utility-maximizing agents interact
with environments, receiving real (or rational) number rewards in response to
actions, and using those rewards to update their behavior. We have argued that
the decision to limit rewards to real numbers is inappropriate in the context
of AGI, because the real numbers have the Archimedean property, which makes it
impossible to use them to accurately portray the value of actions when a task
involves inherently non-Archimedean rewards. Thus, we argue, traditional
RL cannot possibly lead to AGI, because a genuine AGI should have no trouble
at least making attempts to engage in tasks that inherently involve
non-Archimedean structures. We suggested two possible ways
to modify traditional reinforcement learning to fix this bug: switch to
preference-based reinforcement learning, or else generalize reinforcement learning
to allow rewards from a non-Archimedean number system.

\section*{Acknowledgments}

We gratefully acknowledge Brendon Miller-Boldt, Mikhail Katz,
and Stewart Shapiro for comments and feedback.

\bibliographystyle{plain}
\bibliography{arch}
\end{document}
