\documentclass[reqno]{article}
\usepackage{amsthm}
\usepackage{amsfonts}
\usepackage{amsmath}
\usepackage{mathrsfs}
\usepackage{amssymb}
\theoremstyle{definition}
\newtheorem{theorem}{Theorem}
\newtheorem{lemma}[theorem]{Lemma}
\newtheorem{question}[theorem]{Question}
\newtheorem{example}[theorem]{Example}
\newtheorem{proposition}[theorem]{Proposition}
\newtheorem{corollary}[theorem]{Corollary}
\newtheorem{definition}{Definition}

\begin{document}

\title{Archimedean measurements considered poor for non-Archimedean structures}

\author{Samuel A.~Alexander\thanks{Email:
samuelallenalexander@gmail.com}\hphantom{*}\footnote{2010 Mathematics 
Subject Classification: FillThisIn
(Primary), FillThisIn (Secondary)}\\
\emph{The U.S.\ Securities and Exchange Commission}}
\date{2020}
\maketitle

\begin{abstract}
    Fill this in.
\end{abstract}

\section{Introduction}


\begin{lemma}
(The Archimedean Property of the Real Numbers)
If $x_1,x_2,\ldots$ is an infinite sequence of real numbers,
where each $x_{i+1}$ is significantly larger than $x_i$---by which
we mean that $x_{i+1}\geq x_i+1$---then, for every real number $y$,
there exists some $i$ such that $x_i>y$.
\end{lemma}

\begin{proof}
If not, then there is some $y$ such that $y\geq x_i$ for all $i$.
Thus, $X=\{x_1,x_2,\ldots\}$ has an upper bound. By the completeness
of $\mathbb R$, $X$ must have a least upper bound $z\in\mathbb R$.
Since $z$ is the least upper bound for $X$, $z-1$ is not an upper bound
for $X$, so there is some $i$ such that $x_i>z-1$.
By assumption, $x_{i+1}>x_i+1$, so $x_{i+1}>z$, contradicting the choice
of $z$.
\end{proof}


\end{document}
